\documentclass[11pt, a4paper]{article}
\usepackage{amsfonts, amsmath, hanging, hyperref, natbib, parskip, times}
\usepackage[pdftex]{graphicx}
\usepackage[utf8]{inputenx}
\hypersetup{
  colorlinks,
  linkcolor=blue,
  urlcolor=blue
}

\let\section=\subsubsection
\newcommand{\pkg}[1]{{\normalfont\fontseries{b}\selectfont #1}} 
\let\proglang=\textit
\let\code=\texttt 
\renewcommand{\title}[1]{\begin{center}{\bf \LARGE #1}\end{center}}
\newcommand{\affiliations}{\footnotesize}
\newcommand{\keywords}{\paragraph{Keywords:}}

\setlength{\topmargin}{-15mm}
\setlength{\oddsidemargin}{-2mm}
\setlength{\textwidth}{165mm}
\setlength{\textheight}{250mm}

\begin{document}
\pagestyle{empty}

\title{\pkg{hyperSpecGUI}: \\Graphical Interaction in Spectroscopic Data Analysis}

\begin{center}
  \textbf{Sebastian Mellor$^1$, Claudia Beleites$^{2,^\star}$, Colin Gillespie$^1$, Christoph Krafft$^{2}$,
    and Jürgen Popp$^{2,3}$}
\end{center}

\begin{affiliations}
  1. School of Mathematics \& Statistics, Newcastle University, Newcastle/UK.\\[-2pt]
  2. Institute of Photonic Technology, Jena/Germany.\\[-2pt]
  3. Institute of Physical Chemistry and Abbe Center of Photonics, University Jena/Germany.\\[-2pt]
  $^\star$Corresponding author:
  \href{mailto:Claudia.Beleites@ipht-jena.de}{Claudia.Beleites@ipht-jena.de}
\end{affiliations}

\keywords spectroscopy; Google Summer of Code; GUI.

\vskip 0.8cm

Spectroscopic data analysis includes highly visual tasks that need graphical
user interaction. We present \pkg{hyperSpecGUI}, a companion package enhancing
\pkg{hyperSpec}. Both packages are hosted at
\href{http://hyperspec.r-forge.r-project.org/}{hyperspec.r-forge.r-project.org}.

% In contrast to the design of many existing spectroscopic data analysis GUIs, we
% do not replace the command line interface with a graphical user interface (GUI).
% Instead we supplement the scripting facility. 

The \pkg{hyperSpecGUI} packages follows the ``specialise on a single thing''
paradigm, as opposed to implementing an integrated development environment (IDE)
or all-(spectroscopic)-purpose GUI. Visual interaction is provided by small,
specialised GUIs (applets, dialogs) that are called as functions. This allows
full GUI interaction with scripts and easy integration with IDEs. In particular,
this allows:
\begin{description}
\item[Fast work-flow creation:] rather than a single GUI, we use a combination
  of small, bespoke GUIs. This allows swapping between command line and applets.
\item[Flexibility:] the package is extensible and easily adapted to new and
  unforeseen tasks.
\item[Compatibility with batch processing:] spectroscopic data sets are large
  with data analysis often including high performance computations on remote
  servers that do not provide graphical interaction. Switching back and forth
  between GUI-based and command line based work/batch processing are smooth and
  easy.
\item[Reproducibility:] interactive tasks are automatically recorded, allowing
  literate programming techniques to be used, e.g. Sweave.
\end{description}

\emph{Interactive spike correction example:} we present a GUI for spike
filtering of Raman data, demonstrating the wrapping of this GUI with existing
IDEs/editors.

When cosmic rays hit the detector, spikes are observed in the Raman spectra.
Several strategies to identify these artifacts exist and work well with high,
sharp spikes. However, manual control and adjustment of the results is necessary
as broader artifacts may be confused with sharp Raman bands and vice versa. Also
the borders of broader artifacts are difficult to detect.

\emph{Acknowledgements.~} SM's work is funded by the Google Summer of Code 2011.
CB acknowledges funding by the European Union via the Europäischer Fonds für
Regionale Entwicklung (EFRE) and the ``Thüringer Ministerium für Bildung,
Wissenschaft und Kultur'' (Project: B714-07037).

\end{document}
