\usepackage[T1]{fontenc}
\usepackage[utf8]{inputenc}

\setlength{\parskip}{\medskipamount}
\setlength{\parindent}{0pt}
\usepackage{color}
\usepackage{babel}

\usepackage{array}
\usepackage{textcomp}
\usepackage{url}
\usepackage{amsmath}
\usepackage{graphicx}
\usepackage[numbers]{natbib}
\usepackage[unicode = true,
            bookmarks = true,
            bookmarksnumbered = false,
            bookmarksopen = false,
            breaklinks = false,
            pdfborder = {0 0 1},
            backref = false,
            colorlinks = true,
            citecolor=green!50!black,        % color of links to bibli
            filecolor=blue!50!black,      % color of file links
            urlcolor=blue!50!black,           % color of external links
            linkcolor=blue!50!black          % color of internal links
           ]{hyperref}
\usepackage{hyphenat}

\AtBeginDocument{
  	\setlength{\parskip}{\medskipamount}
	\setlength{\parindent}{0pt}
   \fvset{listparameters={\setlength{\topsep}{0pt}}}
   \renewenvironment{Schunk}{\vspace{\topsep}\begin{small}}{\end{small}\vspace{\topsep}
   }
}

% my preferred packages
\usepackage{xspace}
\usepackage{tikz}
\usepackage{subfig}
\usepackage{booktabs}

\usepackage{hyphenat}
\usepackage{fancyvrb}
%\usepackage{siunitx}
%\usepackage{relsize}

\makeatletter
\@ifundefined{showcaptionsetup}{}{%
 \PassOptionsToPackage{caption=false}{subfig}}
\usepackage{subfig}
\makeatother

% fancy warning box
\newcommand{\warnbox}[2]{
\begin{tikzpicture}
\node [draw=red, very thick, rectangle, rounded corners, inner sep=10pt] (box){%
    \begin{minipage}{\linewidth}
      \vspace{0.5\baselineskip}
      #2
    \end{minipage}
};
\node[fill = red, text = white, right=5mm, rounded corners] at (box.north west)
  {\sffamily\bfseries\large #1};
\end{tikzpicture}%
}

\newcommand{\Rcode}[1]{\texorpdfstring{\nohyphens{\texttt{#1}}}{#1}}
\newcommand{\Robject}[1]{\texorpdfstring{\nohyphens{\texttt{#1}}}{#1}}
\newcommand{\Rcommand}[1]{\texorpdfstring{\nohyphens{\texttt{#1}}}{#1}}
\newcommand{\Rfunction}[1]{\texorpdfstring{\nohyphens{\texttt{#1}}}{#1}}

\newcommand{\Rfunarg}[1]{\texorpdfstring{\nohyphens{\textit{#1}}}{#1}}
\newcommand{\Rpackage}[1]{\texorpdfstring{\nohyphens{\textit{#1}}}{#1}}
\newcommand{\Rmethod}[1]{\texorpdfstring{\nohyphens{\textit{#1}}}{#1}}
\newcommand{\Rclass}[1]{\texorpdfstring{\nohyphens{\textit{#1}}}{#1}}

\newcommand{\mFun}[1]{\marginpar{\scriptsize \Rfunction{#1}}}

\newcommand{\phy}{\texorpdfstring{\nohyphens{\textit{hyperSpec}}}{hyperSpec}\xspace}
\newcommand{\chy}{\Rclass{hyperSpec}\xspace}

\newcommand{\eg}{e.\,g.\xspace}
\newcommand{\ie}{i.\,e.\xspace}

\newcommand{\mum}[1]{\ensuremath{#1\;}\textmu m\xspace}
\newcommand{\rcm}[1]{\ensuremath{#1\;\mathrm{cm^{-1}}}\xspace}

\newcommand{\R}{\texorpdfstring{\texttt{R}}{R}\xspace}

\author{Claudia Beleites (\url{cbeleites@units.it})\\
CENMAT, DMRN, University of Trieste}

\SweaveOpts{pgf = FALSE, eps = FALSE, external = FALSE, pdf = TRUE, ps = FALSE}
\SweaveOpts{width=6,height=3}
\SweaveOpts{prefix.string=fig/fig}
\SweaveOpts{keep.source = TRUE, strip.white=TRUE}
\AtBeginDocument{
\setkeys{Gin}{width = .5\textwidth}
}
