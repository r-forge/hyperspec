\documentclass[a4paper, 10pt, smallheadings, DIV15]{scrartcl}
\usepackage[utf8]{inputenc}
\usepackage{tikz}
\pgfrealjobname{spcbin}
\usepackage[labelformat=parens, font=footnotesize]{subfig}
\usepackage{amsmath}
\usepackage{upgreek}
\usepackage{xspace}
\usepackage{hyphenat}
\usepackage{fancyvrb}

\newcommand{\code}[1]{\nohyphens{\texttt{#1}}\xspace}

\newcommand{\R}{\proglang{R}\xspace}
\newcommand{\hy}{\code{hyperSpec}\xspace}

\newcommand{\mum}[1]{\ensuremath{#1\;\upmu}m\xspace}
\newcommand{\rcm}[1]{\ensuremath{#1\;\mathrm{cm^{-1}}}\xspace}

\author{C.~Beleites,\\DMRN, Università degli Studi di Trieste, Trieste/I}
\title{Performance of hyperSpec::spc.bin}

\fvset{listparameters={\setlength{\topsep}{0pt}}}
\renewenvironment{Schunk}{\vspace{\topsep}}{\vspace{\topsep}}

\usepackage{/usr/lib/R/share/texmf/Sweave}
\begin{document}


 

\setkeys{Gin}{width = \textwidth}
\maketitle
\begin{Schunk}
\begin{Sinput}
> library(hyperSpec)
\end{Sinput}
\end{Schunk}

This vignette discusses \hy's \code{spc.bin} function, and runs checks on the correctness of the 
three possible values for \code{na.rm}.   

\section*{Syntax \& Parameters}  
\code{spc.bin (spc, by = stop ("reduction factor needed"), na.rm = TRUE, ...)}

\begin{labeling}{\code{na.rm}:~~~}
\item[\code{spc}:] \code{hyperSpec} object 
\item[\code{by}:] the number of data points to bin into a new data point.
\item[\code{na.rm}:] one of \code{0} or \code{FALSE}, \code{1} or \code{TRUE}, or \code{2}.\\ 
\code{0} or \code{FALSE} has the usual meaning: any \code{NA} in \code{spc} will lead to a \code{NA} in the result.\\
for positive values, \code{NA}s are removed. \code{NaN} results if all elements to bin are \code{NA}.\\ 
\code{1} or \code{TRUE}: takes ca. twice the time compared to \code{na.rm = FALSE}. 
For each bin, the not \code{NA} values are summed up, and divided by their number.\\
\code{2}: This is faster than \code{na.rm = 1} if not the spectra matrix does not contain too many \code{NA}s.\\ 
First, the binning is done with \code{na.rm = FALSE}. Then \code{NA}s are replaced using \code{mean (..., na.rm = TRUE)}.
\item[\code{short}, \code{user}, \code{date}:] are handed to \code{logentry}
\end{labeling}

\begin{Schunk}
\begin{Sinput}
> wls <- c(100, 250, 500, 1000, 2500, 5000)
> rows <- c(10, 25, 50, 100, 250, 500, 1000)
> nas <- c(0, 5, 10, 50, 100, 500, 1000, 2500, 5000, 10000, 25000, 
+     50000)
> bys <- c(2, 5, 10, 25, 50)
> alg <- 1:3
> df <- expand.grid(alg = alg, by = bys, na = nas, nrow = rows, 
+     nwl = wls)
> df$time <- NA
> df <- df[df$na <= df$nwl * df$nrow/10, ]
\end{Sinput}
\end{Schunk}



\begin{Schunk}
\begin{Sinput}
> for (i in seq(1, nrow(df), 3)) {
+     spc <- new("hyperSpec", spc = matrix(runif(df[i, "nrow"] * 
+         df[i, "nwl"]), ncol = df[i, "nwl"]))
+     spc@data$spc[sample(df[i, 4] * df[i, 5], df[i, 3])] <- NA
+     df[i, 6] <- system.time(d1 <- spc.bin(spc, df[i, 2], 0))[1]
+     df[i + 1, 6] <- system.time(d2 <- spc.bin(spc, df[i, 2], 
+         1))[1]
+     df[i + 2, 6] <- system.time(d3 <- spc.bin(spc, df[i, 2], 
+         2))[1]
+     stopifnot(sum(is.nan(d2[[]])) == sum(is.na(d2[[]])))
+     stopifnot(sum(is.nan(d3[[]])) == sum(is.na(d3[[]])))
+     stopifnot(all.equal(d1@data$spc[!is.na(d1@data$spc)], d2@data$spc[!is.na(d1@data$spc)]))
+     stopifnot(all.equal(d1@data$spc[!is.na(d1@data$spc)], d3@data$spc[!is.na(d1@data$spc)]))
+     stopifnot(all.equal(d2@data$spc, d3@data$spc))
+ }
\end{Sinput}
\end{Schunk}

\setkeys{Gin}{width = \textwidth}
\begin{Schunk}
\begin{Sinput}
> save(df, file = "df.RData")
> print(summary(lm(time ~ nrow * nwl, data = df[df$alg == 1, ])))
\end{Sinput}
\begin{Soutput}
Call:
lm(formula = time ~ nrow * nwl, data = df[df$alg == 1, ])

Residuals:
      Min        1Q    Median        3Q       Max 
-4.031177 -0.115102 -0.008146  0.026240  9.697823 

Coefficients:
             Estimate Std. Error t value Pr(>|t|)    
(Intercept) 3.659e-03  4.695e-02   0.078    0.938    
nrow        5.005e-04  9.514e-05   5.261 1.59e-07 ***
nwl         5.949e-06  1.809e-05   0.329    0.742    
nrow:nwl    1.139e-06  3.868e-08  29.456  < 2e-16 ***
---
Signif. codes:  0 ‘***’ 0.001 ‘**’ 0.01 ‘*’ 0.05 ‘.’ 0.1 ‘ ’ 1 

Residual standard error: 1.069 on 1946 degrees of freedom
Multiple R-squared: 0.5854,	Adjusted R-squared: 0.5848 
F-statistic:   916 on 3 and 1946 DF,  p-value: < 2.2e-16 
\end{Soutput}
\begin{Sinput}
> summary(lm(time ~ alg, data = df))
\end{Sinput}
\begin{Soutput}
Call:
lm(formula = time ~ alg, data = df)

Residuals:
     Min       1Q   Median       3Q      Max 
-1.49768 -1.08623 -0.76642 -0.01842 29.46758 

Coefficients:
            Estimate Std. Error t value Pr(>|t|)    
(Intercept)  0.63591    0.08661   7.342 2.39e-13 ***
alg          0.29126    0.04009   7.265 4.23e-13 ***
---
Signif. codes:  0 ‘***’ 0.001 ‘**’ 0.01 ‘*’ 0.05 ‘.’ 0.1 ‘ ’ 1 

Residual standard error: 2.504 on 5848 degrees of freedom
Multiple R-squared: 0.008944,	Adjusted R-squared: 0.008774 
F-statistic: 52.78 on 1 and 5848 DF,  p-value: 4.228e-13 
\end{Soutput}
\end{Schunk}


pgfSweave took 0 seconds.
\end{document}
